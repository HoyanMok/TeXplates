\documentclass[openany]{book} 
% \documentclass{ctexbook} 如果用中文
% \documentclass[10pt,a4paper]{ctexart}  字体大小和纸张大小,默认分别为10pt和letterpaper
% 五号 = 10.5pt,小四=12pt,四号=14pt
% 其他可选参量如twocolumn, 两行排版

\usepackage{biblatex} %[style=gb7714-2015]{biblatex} 可以选择样式
\addbibresource{Mathematics.bib} % 把这里改成实际的文件名

% 令参考资料能够加入目录中:
\defbibheading{bibliography}[\bibname]{% 
	% \addcontentsline{toc}{chapter}{参考文献}
	\chapter{#1}% 
	\markboth{#1}{#1}}

\usepackage{imakeidx} %索引
	\makeindex[intoc]
	\newcommand*{\indexbf}[1]{\textit{\textbf{#1}}\index{#1}} % Index for definition
	\newcommand*{\indexfm}[2]{#1\index{  Symbols@Symbols used!#2@$#1$}} % Used Symbol

% 将PATH换成绝对路径 (Windows) 或相对路径 (Mac OS或Linux)
% 使用「/」而不是「\」
\newcommand{\PATH}{PATH/}

% 对目录项等的修改
\renewcommand{\thesection}{\textmd{\S}\arabic{section}}
\renewcommand{\thesubsection}{\arabic{section}.\arabic{subsection}}


% 引用的宏包:
% 宏包的使用, 可以在命令行运行texdoc <宏包名>获得文档
\usepackage{multicol} % 分栏 (全局分栏建议在文档类处设置)
\usepackage{amsmath} % AMS数学标准
	\makeatletter % '@' now normal "letter"
	\@addtoreset{equation}{section} % 每次换section就把equation清零
	\makeatother  % '@' is restored as "non-letter"
	\renewcommand\theequation{\oldstylenums{\arabic{section}}%
					-\oldstylenums{\arabic{equation}}} % 显示为section数-equation数
\usepackage{amssymb} % 数学符号
\usepackage{mathrsfs} % 花体
\usepackage{esint} % 积分
\usepackage{siunitx} % 标准SI数值和单位处理

\usepackage{tikz} % 绘图
\usepackage{float} % 浮动体 (供图片, 表格等) 扩展, 主要用于提供h模式
\usepackage{graphicx} % 插入图片
\usepackage{titlepic}
\usepackage[font=small, skip=5pt]{caption} % 缩小题注字体和题注与图片距离
\usepackage{subcaption} % 子图和子图的题注
\usepackage{svg} % svg位图
\usepackage{wrapfig} % 简单的图文绕排
\usepackage[inline]{enumitem} % 编号
\usepackage{geometry} % 调整页边距
% \geometry{left=1.6cm,right=1.6cm}
\usepackage{xcolor} % 颜色
\usepackage[colorlinks=true,bookmarks=true]{hyperref} % 引用, 交叉引用, 图表等的链接; 生成书签
\hypersetup{linkcolor=[rgb]{1,0.27,0},bookmarksopen = true}% 更多设置请查阅: texdoc hyperref


% 定义一些笔者常用的指令:
\newcommand{\me}{\mathrm{e}} % 自然对数的底
\newcommand{\mi}{\mathrm{i}} % 虚数单位
\newcommand{\dif}{\mathop{}\!\mathrm{d}} % 微分算子d
\newcommand*{\basis}[1]{\hat{\boldsymbol{#1}}} % 基底
\newcommand*{\bv}{\boldsymbol} % 向量加粗

\newcommand*{\diff}[3][1]
{\if#11%
	\frac{\mathrm{d} #2}{\mathrm{d} #3}% 导数\diff{y}{x}
\else%
	\frac{\mathrm{d}^{#1} #2}{\mathrm{d} #3^{#1}}% n阶导数\diff[n]{y}{x}
\fi}
\newcommand*{\pdiff}[3][1]
{\if#11%
	\frac{\partial #2}{\partial #3}% 偏导数\pdiff{y}{x}
\else%
	\frac{\partial^{#1} #2}{\partial #3^{#1}}% n阶偏导数\pdiff[n]{y}{x}
\fi}

% 笔者习惯的运算符:
\DeclareMathOperator{\tg}{tg}
\DeclareMathOperator{\ctg}{ctg}
\DeclareMathOperator{\arctg}{arctg}
\DeclareMathOperator{\sh}{sh}
\DeclareMathOperator{\ch}{ch}
\DeclareMathOperator{\dom}{dom}
\DeclareMathOperator{\ran}{ran}
\DeclareMathOperator{\interior}{int}
% \DeclareMathOperator*{\指令}{显示} 
% 带星号的版本会像\lim一样

% 文章标题页信息:
\title{Title}
\author{ author1\thanks{contact to author1}
	\and
		author2\thanks{contact to author2}}
\date{\today} % 自动生成日期
\titlepic{\includegraphics{\PATH latex-project-logo.pdf}}

\begin{document}
\frontmatter
\maketitle % 打印标题
\chapter{Preface}
Preface.

\addcontentsline{toc}{chapter}{Contents}
\tableofcontents

\mainmatter
\chapter{Test}
\section{\texorpdfstring{\TeX{}plate Usage}{TeXplate Usage}}
\subsection{Compiling}

Change the \verb|PATH/| to the relative path (Mac OS or Linux) or absolute path (Windows) to this file in \verb|\newcommand{\PATH}{PATH/}| in the preamble, and run the following command in your command line:\\[0pt]
\verb|xelatex| filename (.tex can be omitted)\\
\verb|biber| filename (.bib can be omitted)\\
\verb|xelatex| filename\\
\verb|xelatex| filename

\subsection{Examples}

Citation: \cite{Knuth}.

\begin{figure}[h] %
\centering
\begin{subfigure}{.5\textwidth}
	\centering
	\includegraphics[width=6cm]{\PATH latex-project-logo.pdf}
	\caption{Caption}
	\label{subfigure1}
\end{subfigure}%
\begin{subfigure}{.5\textwidth}
	\centering
	\includegraphics[width=6cm]{\PATH latex-project-logo.pdf}
	\caption{Caption}
	\label{subfigure2}
\end{subfigure}
\caption{An example for subfigures}
\end{figure}

An example for \indexbf{cross reference}: Fig~\ref{subfigure1} is the logo for the \LaTeX{} Project.

Inline equation:
\begin{equation}\label{公式的标签}
	\int \indexfm{\diff y x}\,\dif x = y(x) + C.
\end{equation}

Multiline equations and the alignment:
\begin{align}
	\nabla\cdot \bv D &= \rho
	\\
	\nabla \times \bv E & =  - \pdiff{\bv B}{t}
	\\
	\nabla \cdot \bv B &= 0
	\\
	\nabla \times \bv H &= \bv j + \pdiff{\bv D}{t}
\end{align}

\chapter{Test}
\section{Test}
The counter \verb|section| doesn't reset when a new chapter opened.

\appendix
\chapter{Appendix}

\backmatter
\nocite{*} % 这个表示列出所有没有在文中被引用的参考文献
\printbibliography[heading=bibliography, title={Bibliography}]


\printindex
\end{document}